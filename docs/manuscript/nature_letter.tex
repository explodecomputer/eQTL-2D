\documentclass{article}

\title{Evidence for widespread epistasis in the control of human gene expression}


\begin{document}

\maketitle


\begin{abstract}
A long standing question in human genetics is the extent to which epistasis, where one mutation's effect on a trait depends on other mutations present in the genome, contributes to complex traits. Though epistasis is known to arise from artificial gene manipulation studies in model organisms, and examples have been shown in other species, few convincing examples exist for epistasis amongst natural polymorphisms in human traits. Its absence from empirical findings could be due to its unimportance in the genetic control of complex traits, or it may be too technically difficult to detect due to statistical power and computational issues. Here we show that, using advanced computation techniques and a gene expression study design where any effects are expected to be large, abundant evidence for epistasis is found. In a cohort of 842 individuals with data on 7339 gene expression levels in whole blood, we found that after stringent correction for multiple testing the expression of 249 genes is controlled by 549 significant pairwise epistatic interactions. We attempted replication in two independent datasets and 421 show evidence of significance in at least one dataset. Several genes are controlled by multi-locus epistatic interactions whereby one cis-acting single nucleotide polymorphism (SNP) is modulated by several trans-acting SNPs, for example MBNL1 is controlled by an additive effect at rs13069559 which itself is controlled by trans-SNPs on 18 different chromosomes, with nearly identical genotype-phenotype (GP) maps for each cis-trans interaction. This study presents the first strong evidence for the widespread existence of epistasis emerging from natural genetic variation in humans.
\end{abstract}


\section{Main}

The past decade has seen a tremendous amount of activity in mapping genetic polymorphisms that underlie complex traits. Typically, SNPs are treated as contributing linearly, independently, and cumulatively to the mean of a trait and this has been successful in identifying hundreds of causal variants. Yet outside the prism of association studies there is widespread evidence for epistasis, not only at the molecular scale in artificial double gene knockouts in experimental organisms but also at the evolutionary scale in fitness adaptation and speciation. Though its importance is frequently the subject of debate so far there is little convincing empirical evidence for epistasis playing a substantial role in the architecture of complex traits.

The detection of epistasis is hampered by power issues for several reasons, including increased dependence on linkage disequilibrium (LD) between causal SNPs and observed SNPs, increased model complexity in fitting interaction terms, and more extreme significance thresholds to account for increased multiple testing. When genetic effect sizes are small, as is expected in most complex traits of interest, the power to detect epistasis diminishes rapidly. There are two ways to overcome this problem, either use extremely large sample sizes, or use traits that are likely to have large effect sizes. Because our focus was to ascertain the extent to which epistasis exists amongst natural genetic variation we opted for the latter approach and searched for epistasis controlling gene expression levels. Though they number in the thousands, these traits are typically heritable but much less polygenic, thus it is expected that any genetic effects will be relatively large, maximising the chance at detecting epistasis if it should exist.

We searched for pairwise epistasis exhaustively in the BSGS dataset which comprises 842 individuals who are genotyped at 528,509 SNPs and who have gene expression levels measured in whole blood samples for 7339 genes. Recent hardware and software advances made it possible to perform the $1.03 \times 10^{15}$ statistical tests to complete this analysis. We used permutation analysis to calculate an experiment-wide significance threshold of $2.91 \times 10^{-16}$ at the 5\% family-wise error rate (FWER). SNP pairs were modelled for full genetic effects, including marginal additive and dominance at both SNPs plus four interaction terms. Though we could have used a less complex model to improve statistical efficiency, we deemed it important to be agnostic about the type of epistasis that might exist, and therefore chose not to over-parameterise the test. Because there are many large marginal effects present in these data it was necessary to perform several filtering steps to exclude SNP pairs that were significant due to marginal effects alone. All SNP pairs with LD $r^2 > 0.1$ were removed, and were required to have at least 5 data points in all nine genotype classes. If multiple SNP pairs were present on the same chromosomes for a particular expression trait then only the sentinal SNP pair was retained. Finally, a nested test contrasting the full genetic model against the marginal additive and dominance model was performed for each remaining SNP pair, resulting in 549 significant interactions after Bonferroni correction for multiple testing.

The significant SNP pairs were carried forward for replication in two independent datasets that used the same expression assays for whole blood, Fehrmann ($n=1240$) and EGCUT ($n=891$). There was substantial evidence for the same effects being present in all three datasets. Of the 480 original pairs that passed filtering in Fehrmann and EGCUT, 39 were significant at the Bonferroni level in at least one, and of these 18 were significant in both. The GP maps for these interactions are remarkably similar in all three datasets (Figure \ref{fig:gpmaps}). 

We also attempted replication in a third dataset, CHDWB, but only 185 of the SNP pairs passed filtering because the sample size was small ($n=139$), and due to insufficient power there was no evidence for replication.


\section{Notes}

Results


- Hundreds of significant pairwise interactions controlling gene expression replicate in two independent studies



- The genotype-phenotype maps for the most significant replications are consistent in all three datasets



- The proportion of the variance that is due to interactions is generally smaller than additive. There is an ascertainment bias in that there are no genetic interactions strong enough to be detected on their own, and large additive effects are required to push them above the threshold



- There are several examples of gene expression being controlled by a cis-SNP which in turn is modulated by multiple trans-SNPs. The effects are consistent 






\section{Methods summary}





\end{document}