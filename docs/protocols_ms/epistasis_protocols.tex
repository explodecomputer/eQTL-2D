\documentclass{article}
\usepackage[superscript,biblabel]{cite}
\usepackage{graphicx}
\usepackage{float}
\usepackage{threeparttable}
\usepackage{threeparttablex}
\usepackage{authblk}
\usepackage{multirow}
\usepackage{longtable}
\usepackage{lscape}

\makeatletter
\renewcommand\AB@affilsepx{. \protect\Affilfont}
\makeatother

%\graphicspath{{../../analysis/images/}}
\bibliographystyle{naturemag}

\title{How to find epistasis, Bitches!}
\date{}

\author[1,2]{Gibran Hemani}
\author[1,2]{Joseph E Powell}

\affil[1]{University of Queensland Diamantina Institute, University of Queensland, Princess Alexandra Hospital, Brisbane, Queensland, Australia}
\affil[2]{Queensland Brain Institute, University of Queensland, Brisbane, QLD, Australia}

\begin{document}

\maketitle

\clearpage

\section{Abstract}




\clearpage

\section{Introduction}

Epistasis is the process whereby the effect of one genetic loci on a phenotype is modified by the genotypes carried at other unlinked loci. As a topic in human genetics it has been widely discussed but rarely investigated in depth due to a series of computational and statistical challenges. Recently, a number of these challenges have been overcome and we have recently shown that through careful application of these methods (wide scale) epistasis can be detected for gene expression phenotypes in humans (ref). [I think we should try and say something about the processes needed to interoperate the output of a epistasis analysis]  


\subsection{Challenges}

[Initially outline the challenges in the introduction. These can be address through the methods section] \\
\newline
1. Computational \\ 
\newline
[I think it would be worth re-discussing some of the computational problems of 2d scans. and highlighting the software + hardware solutions] \\
\newline
One of the first obstacles that make epistatic searches difficult is the computational demands of the statistical analyses. When searching for independent additive effects, as is done for the majority of GWA studies, each SNP is tested for association with the phenotype. However, in order to most powerfully identify epistatic effects, the search must be increased to multiple dimensions (ref). Here the scale of the computational demand increases by $~(n^x)/2$ where $n$ is the numer of snps and $x$ is the number of epistatsis dimensions fitted. For example, testing for 2 loci interactions using SNP data from a 1 million SNP chip would require $1000000 * 999999/2 \approx 5e11$ individual tests. \\     
\newline 
2. Model choice \\
\newline
From a 2d 8df biallelic SNP model there are 4 epistatic variance components; additive x additive, additive x
dominance, dominance x additive and dominance x dominance. Within the literature there has been considerable discussion as to the likely degree of epistatsis variance components and whether statistical power could be increased by parameterizing models for only a subset. We have shown then estimates of parameters repressenting the 4 variance components are proportionally represented amongst the significant epistatis pairs. This implys a full model is preferential, particularly if we are agonostic about the mechanism by which epistasis might arise. \\  
\newline
3. Statistical \\
\newline
A complete exhaustive scan of $m$ phenotypes and $n$ SNPs comprises of $((n*(n-1))^2/2)*m$ $8df$ F-tests. Given the high correlation structure inherent in genotype data as well as between multiple phenotypes, choices regarding multiple testing need to be carefully considered. [what can we say about tails of the distributions?]
[Perhaps this is a good point to outline out a 2-step procedure where main effects are initially corrected for] \\
The 8df F-test includes parameters for main (additive) effects of the two SNPs. Therefore it is important to test that test statistic isn't driven my marginal effects - something that is quite likely for a trait such as gene expression. Thus one should really consider a nested test.  \\
\newline
4. Interpretation / filtering \\
\newline
Things to consider and address during the analysis and output, particularly given    
\newline
\section{Methods}

I think one approach for this paper (which is in keeping with other Nature Protocol manuscripts) is to go through the whole study design and show how we have done things - discussing choices in terms of the challenges mentioned above. For many steps we can show example (pseudo) code or give it in the supporting material  

\subsection{Stages of an epistasis analysis}

\subsubsection{Data}

QC \\
Normalisation (distributon with respect to models used) \\
Statistical model choice \\

\subsubsection{Computational / Statistics}



\subsubsection{Multiple testing}











\end{document}

