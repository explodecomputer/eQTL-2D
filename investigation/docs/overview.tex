%%%%%%%%%%%%%%%%%%%%%%%%%%%%%%%%%%%%%%%%%%%%%%%%%%%%%%%%%%%%%%%%%%%%%%
%
% Investigation into the architecture of epistasis
%
%%%%%%%%%%%%%%%%%%%%%%%%%%%%%%%%%%%%%%%%%%%%%%%%%%%%%%%%%%%%%%%%%%%%%%
%\title{Gene expression underlying Endo susceptability}
%
%%% Preamble

\documentclass[paper=a4, fontsize=11pt]{scrartcl}					% Article class of KOMA-script with 11pt font and a4 format
\usepackage[T1]{fontenc}
\usepackage{fourier}
\usepackage{float}
\usepackage{placeins}
\usepackage[english]{babel}											% English language/hyphenation
\usepackage[protrusion=true,expansion=true]{microtype}				% Better typography
\usepackage{amsmath,amsfonts,amsthm}								% Math packages
\usepackage[pdftex]{graphicx}										% Enable pdflatex
\usepackage{url}
\usepackage{multicol}
\usepackage{lscape}
\usepackage{parskip}



%%% Custom sectioning (sectsty package)
\usepackage{sectsty}												% Custom sectioning (see below)
\allsectionsfont{\centering \normalfont\scshape}					% Change font of al section commands

%%% Custom headers/footers (fancyhdr package)
\usepackage{fancyhdr}
\pagestyle{fancyplain}
\fancyhead{}														% No page header
\fancyfoot[L]{}														% Empty 
\fancyfoot[C]{}														% Empty
\fancyfoot[R]{\thepage}												% Pagenumbering
\renewcommand{\headrulewidth}{0pt}									% Remove header underlines
\renewcommand{\footrulewidth}{0pt}									% Remove footer underlines
\setlength{\headheight}{13.6pt}
\setlength{\parindent}{0pt}

\usepackage{tabularx,ragged2e,booktabs,caption}
\newcolumntype{C}[1]{>{\Centering}m{#1}}
\renewcommand\tabularxcolumn[1]{C{#1}}

%%% Equation and float numbering
\numberwithin{equation}{section}									% Equationnumbering: section.eq#
\numberwithin{figure}{section}										% Figurenumbering: section.fig#
\numberwithin{table}{section}										% Tablenumbering: section.tab#

%%% Referencing
\bibliographystyle{naturemag}


%%% Maketitle metadata
\newcommand{\horrule}[1]{\rule{\linewidth}{#1}} 					% Horizontal rule

\title{
		%\vspace{-1in} 	
		\usefont{OT1}{bch}{b}{n}
		\normalfont \normalsize \textsc{University of Queensland} \\ [25pt]
		\horrule{0.5pt} \\[0.4cm]
		\huge Epistasis follow-up \\[0.3cm]
        \huge *** summary *** \\
		\horrule{2pt} \\[0.5cm]
		\date{28th Nov 2014}
}

\begin{document}

%%%%%%%%%%%%%%%%%%%%%%%%%%%%%%%%%%%%%%%%%%%%%%%%%%%%%%%%%%%%%%%%%%%%%%%%%%%%%%%%%%%%%
%%%%%%%%%%%%%%%%%%%%%%%%%%%%%%%%%%%%%%%%%%%%%%%%%%%%%%%%%%%%%%%%%%%%%%%%%%%%%%%%%%%%%
%%%%%%%%%%%%%%%%%%%%%%%%%%%%%%%%%%%%%%%%%%%%%%%%%%%%%%%%%%%%%%%%%%%%%%%%%%%%%%%%%%%%%


%%% Begin document
\maketitle


%%%%%%%%%%%%%%%%%%%%%%%%%%%%%%%%%%%%%%%%%%%%%%%%%%%%%%%%%%%%%%%%%%%%%%%%%%%%%%%%%%%%%
%%%%%%%%%%%%%%%%%%%%%%%%%%%%%%%%%%%%%%%%%%%%%%%%%%%%%%%%%%%%%%%%%%%%%%%%%%%%%%%%%%%%%
%%%%%%%%%%%%%%%%%%%%%%%%%%%%%%%%%%%%%%%%%%%%%%%%%%%%%%%%%%%%%%%%%%%%%%%%%%%%%%%%%%%%%

\newpage
\section{Analysis summary}
\noindent


Original discovery analysis identified 501 interactions comprising of 781 unique SNPs and 238 genes (probes). These were 26 cis-cis, 462 cis-trans and 13 trans trans. The majority of our discovery interactions were composed of one SNP that was significantly associated with the gene expression level in the discovery data set, and one SNP that had no previous association (439 out of 501, Methods). Only nine interactions were between SNPs that both had known main effects, whereas 64 were between SNPs that had no known main effects.  

The following analyses have been conducted; \\

1. Determining the empirical p-values for each of the 501 interactions \\

The initial analysis used F-tests and some simulation work to determine the expected Type 1 error rate in the 1st stage of the discovery process. Subsequent simulations and theoretical calculations have suggested that this is not correct when there is a large main effect and / or in the presence of LD. \\

a. Identify which of the two snps in the original epistasis pair has the largest additive effect. Then, treating that as a fixed SNP, perform a genome-wide analysis using the 8df and 4df epistasis model. Use the interaction p-values to determine the empirical distribution of null p-values. This has been completed after filtering all SNPs on the fixed SNPs chromosome as well as +/- 5MB around the position of the original second snp.  \\

b. Identify the largest additive eQTL for the probe (irrespective of effect size). Regress out the effect of the additive loci and use the adjusted phenotype for 8df and 4df model analysis. Results reported are the 8df and 4df of original pair and the empirical p-values from genome-wide analysis fitting each of the two snps as fixed. As before these are determined by filtering out the snps on the same chromosome as the original fixed snp and also within +/- 5MB of the second snp. \\

\vspace{1cm}

2. Prediction \\

Of the 501 snp pairs, 484 have both snps in the EGCUT data. Most of the Inchianti snps need to be imputed, but we expect most to pass filtering. For pairs without and Inchiamti snp I propose using the snp with the largest additive effect in the egcut data. \\

For each pair; \\

a. Predict the phenotype in egcut data using a predictor with effects estimated from \\

4df model (estimated in BSGS) \\
8df model (estimated in BSGS) \\
1df model (estimated in BSGS using the Inchianti snp) \\








%%%%%%%%%%%%%%%%%%%%%%%%%%%%%%%%%%%%%%%%%%%%%%%%%%%%%%%%%%%%%%%%%%%%%%%%%%%%%%%%%%%%%
%%%%%%%%%%%%%%%%%%%%%%%%%%%%%%%%%%%%%%%%%%%%%%%%%%%%%%%%%%%%%%%%%%%%%%%%%%%%%%%%%%%%%
%%%%%%%%%%%%%%%%%%%%%%%%%%%%%%%%%%%%%%%%%%%%%%%%%%%%%%%%%%%%%%%%%%%%%%%%%%%%%%%%%%%%%

\newpage
\section{Results}


%%%%%%%%%%%%%%%%%%%%%%%%%%%%%%%%%%%%%%%%%%%%%%%%%%%%%%%%%%%%%%%%%%%%%%%%%%%%%%%%%%%%%
%%%%%%%%%%%%%%%%%%%%%%%%%%%%%%%%%%%%%%%%%%%%%%%%%%%%%%%%%%%%%%%%%%%%%%%%%%%%%%%%%%%%%
%%%%%%%%%%%%%%%%%%%%%%%%%%%%%%%%%%%%%%%%%%%%%%%%%%%%%%%%%%%%%%%%%%%%%%%%%%%%%%%%%%%%%



%%%%%%%%%%%%%%%%%%%%%%%%%%%%%%%%%%%%%%%%%%%%%%%%%%%%%%%%%%%%%%%%%%%%%%%%%%%%%%%%%%%%%
%%%%%%%%%%%%%%%%%%%%%%%%%%%%%%%%%%%%%%%%%%%%%%%%%%%%%%%%%%%%%%%%%%%%%%%%%%%%%%%%%%%%%
%%%%%%%%%%%%%%%%%%%%%%%%%%%%%%%%%%%%%%%%%%%%%%%%%%%%%%%%%%%%%%%%%%%%%%%%%%%%%%%%%%%%%


\section{Methods}



%%% End document
\end{document}



